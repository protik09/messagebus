%%% Template originaly created by Karol Kozioł (mail@karol-koziol.net) and
%modified for ShareLaTeX use

\documentclass[a4paper,11pt]{article}

\usepackage[T1]{fontenc}
\usepackage[utf8]{inputenc}
\usepackage{graphicx}
\usepackage{xcolor}

\renewcommand\familydefault{\sfdefault}
\usepackage{tgheros}
% \usepackage[defaultmono]{droidmono}

\usepackage{amsmath,amssymb,amsthm,textcomp}
\usepackage{enumerate}
\usepackage{multicol}
\usepackage{tikz}

\usepackage{geometry}
\geometry{left=25mm,right=25mm,% bindingoffset=0mm, top=20mm,bottom=20mm}


\linespread{1.3}

\newcommand{\linia}{\rule{\linewidth}{0.5pt}}

% custom theorems if needed
\newtheoremstyle{mytheor}
    {1ex}{1ex}{\normalfont}{0pt}{\scshape}{.}{1ex} {{\thmname{#1
    }}{\thmnumber{#2}}{\thmnote{ (#3)}}}

\theoremstyle{mytheor}
\newtheorem{defi}{Definition}

% my own titles
\makeatletter
\renewcommand{\maketitle}{
\begin{center}
\vspace{2ex}
{\huge \textsc{\@title}}
\vspace{1ex}
\\
\linia\\
\@author \hfill \@date
\vspace{4ex}
\end{center}
}
\makeatother
%%%

% custom footers and headers
\usepackage{fancyhdr}
\pagestyle{fancy}
\lhead{}
\chead{}
\rhead{}
\lfoot{Assignment \textnumero{} 7}
\cfoot{}
\rfoot{Page \thepage}
\renewcommand{\headrulewidth}{0pt}
\renewcommand{\footrulewidth}{0pt}
%

% code listing settings
\usepackage{listings}
\lstset{language=c++, basicstyle=\ttfamily\small, aboveskip={1.0\baselineskip},
    belowskip={1.0\baselineskip}, columns=fixed, extendedchars=true,
    breaklines=true, tabsize=4,
    prebreak=\raisebox{0ex}[0ex][0ex]{\ensuremath{\hookleftarrow}}, frame=lines,
    showtabs=false, showspaces=false, showstringspaces=false,
    keywordstyle=\color[rgb]{0.627,0.126,0.941},
    commentstyle=\color[rgb]{0.133,0.545,0.133},
    stringstyle=\color[rgb]{01,0,0}, numbers=left, numberstyle=\small,
    stepnumber=1, numbersep=10pt, captionpos=t, escapeinside={\%*}{*)}}

%%%----------%%%----------%%%----------%%%----------%%%

\begin{document}

\title{Programming 2 Project}

\author{Protik Banerji (s2342898), Dharanish NH (2372525)}

\date{17/04/2020}

\maketitle

\section*{Problem 7.3.1}

\textbf{\textit{What is a priority queue?}}

A priority queue is an advanced version of the queue where we assign priority to
every element in the queue. Elements that have the highest priority are popped
from the queue (dequeued) first. If there are any two elements in the queue with
the same priority they are removed according to their order in queue..

\texttt{Source: https://www.geeksforgeeks.org/priority-queue-set-1-introduction/}

\section*{Problem 7.3.2}
\textbf{\textit{Why would the writer of the code of MessageBus.h have chosen for storage of TextMessage*, instead of TextMessage?}}

In general it is always better to call data by the pointer of the variable.When
we have large value for a variable. As it avoids creating multiple copies of
variable while performing some operation and also reduce the time in fetching
variables that are large.Pointer also helps in referring the same space in
memory form multiple locations. Polymorphism means the some code or operations
or objects behave differently in different contexts.So it would be very useful
to point the location with pointers of different data types based on the
contexts.Object slicing happens when a derived class object is assigned to base
class and the additional attributes of derived class are sliced off to form base
class object.To overcome this we can use pointers to base is required to be able
to refer to an object of any type derived from base and still retain the correct
type for the derived object.Also we can virtual functions to overcome object
slicing.The use of implementing Textmessage* can be clearly understood in
challenge 2.

\texttt{Source:https://www.geeksforgeeks.org/object-slicing-in-c/}

\section*{Problem 7.3.3}
\textbf{\textit{Explain how TextMessageCompare is used in the priority queue. Note that this is an example of operator overloading.}}

In c++ there is a feature called operator overloading which means that user can
redefine the meaning of the operator for user defined types.The operator used
here is () which allows to use any type and number of data.TextMessagecompare is
used priority in priority queue.The operator will when the message is being
pushed into queue. While pushing its going to compare the priority of the nodes
and rearrange the queue.

\texttt{Source: https://www.learncpp.com/cpp-tutorial/99-overloading-the-parenthesis-operator/}

\section*{Problem 7.3.4}
\textbf{\textit{One could argue that code is missing as we are allocating memory from the heap (that is,we use new). Explain why this is a problem.}}

By using \textbf{new} we are allocating memory from heap which is dynamically
occupying the free memory available.Once the memory is occupied it has to be
deleted if it is not deleted it will cause memory leak. The memory gets
deallocated either by using \textbf{delete} or when the operation ends.If there
is memory leak in an application which keeps on running then eventually all the
memory will be occupied.


\section*{Problem 7.4}

\textbf{\textit{Explain again why using a pointer to instead of an object itself in a storage structure is beneficial.}}

As explained before using pointer to store than object is very beneficial while
storing data. First we restrict duplication of memory calling the object
everytime.Secondly when using pointer the values stored can be easily altered
from any point of the program.Also if the data is large it would be ideal to
fetch the location of data instead copying the whole data into memory,So for
large data it will also help in improving the performance. The memory resources
allocated by object within a block will be there only in the block where it is
created  after the execution  it gets destroyed where else by using pointers we
can assign memory from heap and its deleted only when the delete is called.

\section*{Problem 7.5}

\textbf{\textit{Would you use join or a detach in the start method, considering
the fact that in Simulator.cpp a getline is explicitly used? Explain your
choice.}}

Detach would be used here.If we use join the the calling thread will be stopped
until the called thread finishes the task. By using detach we do not actually
stop the main thread that is the calling thread but it allows both the threads
to execute independently. Since \texttt{getline} is explicitly used the main
thread waits for user to press enter. 

\section*{Problem 7.6.1}
\textbf{\textit{What is a callback function?}}

A callback function is a function that is called at the end of execution of a
function that is \textit{calling} has the \textit{callback function} pass to it
as an argument.

\section*{Problem 7.6.2}
\textbf{\textit{What is a database index and name one advantage and one
disadvantage of its usage; use complexity in your argumentation.}}

A database index is a data structure that is used to speed up access to tables
located in the database, without having to look through all the data in the
database. The index entries are usually generated from the columns of the
database and is usually represented as a B-Tree, who's worst case operations are
\textt{$O(log~n)$}. If we try to do operations \textbf(without the index) then
assuming the database is M-dimensional, we would have database operations of
complexity of \textt{$O(n^M)$}, which is \textbf{significantly} slower than
using an index.

The disadvantage of using a index is the cost of updating and maintaining the
index is a write heavy operation, an overhead which would not exist for
non-indexed databases. (A particular egrigious example being Microsoft Search in
Windows, it uses a significant amount of resources to maintain a searchable
index of all the files and programs on your computer). Also since you now cannot
directly access the database, access time is increased to complexity
\textt{$O(log~n)$}. Without an index access complexity of an M-dimensional
database is \textt{$O(1^M)$}.

\end{document}
